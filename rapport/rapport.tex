\documentclass[11pt]{article}
\usepackage[utf8]{inputenc}
\usepackage[french]{babel}
\usepackage[T1]{fontenc}
\usepackage{verbatim}
\usepackage{graphicx}
\usepackage{amsmath}
\usepackage{eurosym}
\usepackage{textcomp}
\author{Contzen Laurent, Corvalan Gabriel}
\title{NaluRSS}
\date{14 Décembre 2009}

\begin{document}

\begin{titlepage}
\begin{flushleft}
Contzen Laurent \\
Corvalan Gabriel \\
\end{flushleft}
\begin{center}
\vspace{65mm}\LARGE{\textbf{INFO-H-303 - Bases de données} :\\
NaluRSS}
\end{center}
\begin{flushright}
\vspace{70mm}
Année Académique 2009-2010.
\end{flushright}
\end{titlepage}
\tableofcontents
\newpage

\section{Introduction}
Dans le cadre de ce projet du cours de bases de données nous avons du développer un agrégateur de flux rss.
\section{Présentation de NaluRSS}
NaluRSS est un lecteur de flux rss entièrement écrit en xhtml\footnote{eXtendend HyperText Markup Language}/css\footnote{Cascading Style Sheet} et php\footnote{excellente question} et enregistre les informations dans une base de données MySQL. Il a été pensé pour être le plus simple d'utilisation possible pour l'utilisateur.
\section{Script SQL DDL de création de la base de données}

\section{Requêtes demandés}

\section{Instructions d'installation}

\section{Fonctionnalités}

\section{Scénario de démonstration}

\section{Explications diverses}

\section{Conclusion}

\end{document}
